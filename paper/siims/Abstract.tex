% !TEX root = ../Geodesics_BV2.tex

\begin{abstract}
This paper studies the space of $BV^2$ planar curves endowed with the $BV^2$ Finsler metric over its tangent space of displacement vector fields. Such a space is of interest for applications in image processing and computer vision because it enables piecewise regular curves that undergo piecewise regular deformations, such as articulations. The main contribution of this paper is the proof of the existence of the shortest path between any two $BV^2$-curves for this Finsler metric.
%
Such a result is proved by applying the direct method of calculus of variations to minimize the geodesic energy.
This method applies more generally to similar cases such as the space of curves with $H^k$ metrics for $k\geq 2$ integer. This space has a strong Riemannian structure and is geodesically complete. Thus, our result shows that the exponential map is surjective, which is complementary to geodesic completeness in infinite dimensions.
%
We propose a finite element discretization of the minimal geodesic problem, and use a gradient descent method to compute a stationary point of the  energy. Numerical illustrations show the qualitative difference between $BV^2$ and $H^2$ geodesics.

\subjclass{Primary 49J45, 58B05; Secondary 49M25, 68U05.}

\keywords{Geodesics ; Martingale ; $BV^2$-curves ; shape registration}
\end{abstract}