\section{Discretization}
\label{discretization}

This section discretizes Problem~\eqref{grad-rigid} using finite elements in order to calculate numerically the Finsler gradient flow. We define a $n$-dimensional sub-space $\Bb_n \subset \Bb$ of piecewise linear curves. The embedding $\Bb_n \subset \Bb$ defines a natural finite dimensional Riemannian and Finsler structure on $\Bb_n$ inherited from the ones of $\Bb$. This allows us to apply our Finsler gradient flow in finite dimension to approximate the original infinite dimensional Finsler flow. 


%%%%%%%%%%%%%%%%%%%%%%%%%%%%%%%%%%%%%%%%%%%%%%%%%%%%%%%%%%%%%%%%%%%%%%%%%%%%%%%%%%%%%%%
%%%%%%%%%%%%%%%%%%%%%%%%%%%%%%%%%%%%%%%%%%%%%%%%%%%%%%%%%%%%%%%%%%%%%%%%%%%%%%%%%%%%%%%
%%%%%%%%%%%%%%%%%%%%%%%%%%%%%%%%%%%%%%%%%%%%%%%%%%%%%%%%%%%%%%%%%%%%%%%%%%%%%%%%%%%%%%%
\subsection{Finite Elements Spaces}

%%%
\paragraph{Notations.}

In the following, to ease the notation, we identify $\RR^2$ with $\CC$ and $\Circ$ with $[0,1]$ using periodic boundary conditions.  
The canonical inner produced on $\CC^n$ is
\begin{equation}\label{scalarC}
	\dotp{\tilde{f}}{\tilde{g}}_{\mathbb{C}^n} = 
	\sum_{i=1}^n \dotp{\tilde{f}_i}{\tilde{g}_i}
	=
	\sum_{i=1}^n {\rm Real}(\tilde{f}_i \,\overline{\tilde{g}_i}), 
	\,\quad \foralls \tilde{f}, \tilde{g} \in\CC^n \,,
\end{equation}
where we denote by $\overline{\tilde{g}_i}$ the conjugate of $\tilde{g}_i$.

%%%
\paragraph{Piecewise affine finite elements.}

We consider the  space $\PP_{1,n}$ of the finite elements on $[0,1]$ (with periodic boundary conditions) of order one with $n$ equispaced nodes. 
A basis of $\PP_{1,n}$ is defined as 
\begin{align*}
	\xi_i(s) &= \max\left\{0, 1- n\left| s- \frac{i}{n} \right|\right\} \quad  s\in [0,1],\;  \foralls i= 1, ..., n-1 \\
	\xi_n(s) &= \max\left\{0, 1- n\left|s \right|\right\} + \;\max\left\{0, 1- n\left| s- 1 \right|\right\},  \quad s\in [0,1].
\end{align*}
Every  $f\in \PP_{1,n}$ can be written as 
\begin{equation}\label{tangent-discrete}
	f=\sum_{i=1}^n\tilde{ f}_i \, \xi_i\,,\quad \tilde{f}_i\in\CC\,
\end{equation}
with $\tilde{f}_i=f(i/n)\in\CC$ for every $i$. We denote by $\tilde{f}=(\tilde{f}_1,..., \tilde{f}_n)\in \CC^n$ the coordinates of $f$ with respect to  the basis $\{\xi_i\}_{i=1,...,n}$. 
Remark that there exists a bijection between $\PP_{1,n}$ and $\CC^n$, defined by the following operator
\begin{equation}\label{bijection}
	P_1\; : \tilde{f}=(\tilde{f}_1,..., \tilde{f}_n)\in \CC^n 
	\;\;\mapsto\;\; 
	P_1(\tilde{f})=  \; f\in \PP_{1,n} \mbox{\;\;s.t.\;\;} f=\sum_{i=1}^n\tilde{f}_i \, \xi_i\,.
\end{equation}
The forward and backward finite differences operators are defined as
\begin{equation}\label{delta}
\begin{array}{ll}
	\Delta^+: \CC^n\rightarrow  \CC^n \;,& \quad \Delta^+( \tilde{f})_i = n(\tilde{f}_{i+1}-\tilde{f}_i)\,,\\
	\Delta^-: \CC^n\rightarrow  \CC^n \;, &\quad \Delta^-( \tilde{f})_i = n(\tilde{f}_i -\tilde{f}_{i-1})\,,
\end{array}
\end{equation} 

%%%
\paragraph{Piecewise constant finite elements.}

% \newcommand{\myI}{\mathbbm{1}}
\newcommand{\myI}{\mathbb{I}}

For every $f\in \PP_{1,n}$,~\eqref{tangent-discrete} implies that  first derivative $\frac{\d f}{\d s}$ belongs to $\PP_{0,n} \subset  BV([0,1], \RR^2)$, where $\PP_{0,n}$ is the class of the piecewise constant functions with $n$ equispaced  nodes. A basis of $\PP_{0,n}$ is defined by 
\begin{align*}
\zeta_i(s) =\myI{}_{[\frac{i}{n},\frac{i+1}{n}]}(s) \quad  \foralls i= 1, ..., n-1 \,,\quad 
\zeta_n(s) =\myI{}_{[0,\frac{1}{n}]}(s) \,,
\end{align*}
where $\myI_A$ is the characteristic function of a set $A$, and with $s\in [0,1]$. Then, the first derivative of $f$ can be written as 
\begin{equation}\label{first-derivative-approx}
\frac{\d f}{\d s} = \sum_{i=1}^n \Delta^+(\tilde{f})_i\zeta_i\,.
\end{equation}
We finally define the following bijection between $\PP_{0,n}$ and $\CC^n$:
\begin{equation}\label{bijection2}
P_0\; : \; \tilde{f}=(\tilde{f}_1,..., \tilde{f}_n)\in \CC^n\;\;\mapsto\;\; P_0(\tilde{f})= f\in \PP_{0,n}   \mbox{\;\;s.t.\;\;} f=\sum_{i=1}^n\tilde{ f}_i\zeta_i\,.
\end{equation}

%%%%%%%%%%%%%%%%%%%%%%%%%%%%%%%%%%%%%%%%%%%%%%%%%%%%%%%%%%%%%%%%%%%%%%%%%%%%%%%%%%%%%%%
%%%%%%%%%%%%%%%%%%%%%%%%%%%%%%%%%%%%%%%%%%%%%%%%%%%%%%%%%%%%%%%%%%%%%%%%%%%%%%%%%%%%%%%
%%%%%%%%%%%%%%%%%%%%%%%%%%%%%%%%%%%%%%%%%%%%%%%%%%%%%%%%%%%%%%%%%%%%%%%%%%%%%%%%%%%%%%%
\subsection{Finite Element Spaces of Curves} 

%%%
\paragraph{Discretized curves.} 

The discrete space of curves is defined as $\Bb_n= \PP_{1,n} \subset \Bb$ and every curve $\GA\in\Bb_n$ can be written as
\begin{equation}\label{gamma-coord}
	\GA=\sum_{i=1}^n \tGa_i\xi_i\,,\quad \tGa_i=\GA(i/n)\in \CC\,
\end{equation}
where the vector $\tGa=P_1^{-1}(\GA) = (\tGa_1, ..., \tGa_n)\in \CC^n$ contains the coefficients of $\GA$ in the finite element basis. By~\eqref{first-derivative-approx} the tangent and normal vectors~\eqref{def-tangente} to $\GA\in \Bb_n$ are computed as 
\begin{equation}\label{tan-norm}
\tgam= \sum_{i=1}^n\frac{\Delta^+ (\tGa)_i}{|\Delta^+ (\tGa)_i|}\zeta_i  \;,\;\; \ngam(i)= \rot{\tgam(i)}~,
\end{equation}
where $\rot{(x,y)}=(-y,x)$ for all $(x,y) \in \RR^2$.
In particular we have 
\begin{equation}\label{arc-length}
	\frac{\d\GA}{\d s} = \sum_{i=1}^n \Delta^+(\tGa)_i\zeta_i\,.
\end{equation}


%%%
\paragraph{Discretized tangent spaces.}  

For every  $\GA\in\Bb_n$, the discrete tangent space to $\Bb_n$ at $\GA$ is defined as  $T_{\GA} \Bb_n = \Bb_n$ equipped with the  inner product $\dotp{\cdot}{\cdot}_{H^1(\GA)}$. Every vector field $\Phi\in T_\GA \Bb_n$ can be written as
\begin{equation}\label{representation-phi}
\Phi=\sum_{i=1}^n \tPhi_i\xi_i\,,\quad \tPhi_i=\Phi(i/n)\in \CC\,
\end{equation}
where $\tPhi=(\tPhi_1, ..., \tPhi_n)\in \CC^n$ are the coordinates of $\Phi$ with respect to the basis of $\PP_{1,n}$. 

By identifying every vector field $\Phi \in T_\GA\Bb_n$ with its coordinates $\tPhi$, the tangent space can be identified with  $\CC^n$. 
In particular we have 
\begin{equation}\label{arc-length-2}
	\frac{\d\Phi}{\d \Gamma} = \sum_{i=1}^n \frac{\Delta^+(\tilde{\Phi})_i}{|\Delta^+(\tilde{\GA})_i|}\zeta_i\,.
\end{equation}
Moreover, $\CC^n$ can be equipped with the following Riemannian metric:
 \begin{defn}[\BoxTitle{Discrete inner product}]
 We define $\ell^2(\tilde{\GA})$ and  $h^1(\tilde{\GA})$ as the set $\CC^n$ equipped with the following inner products respectively
 \begin{equation}\label{scal-prod-coord} 
	\dotp{\tPhi}{\tilde{ \Psi}}_{\ell^2(\tilde{\GA})} = 
	\dotp{P_1(\tPhi)}{P_1(\tPsi)}_{L^2(\GA)}  \,,
\end{equation} 
\begin{equation}\label{scal-prod-coord-sob} 
	\dotp{\tPhi}{\tilde{ \Psi}}_{h^1(\tilde{\GA})} = 
	\dotp{P_1(\tPhi)}{P_1(\tPsi)}_{H^1(\GA)}  \,. 
\end{equation} 
 \end{defn}

We now give the explicit formulas for the products~\eqref{scal-prod-coord} and \eqref{scal-prod-coord-sob}, which are  useful for computational purposes. 

Proposition~\eqref{prod-scla-cont} details the relationship between the product~\eqref{scal-prod-coord} and the canonical inner product on $\CC^n$ defined by~\eqref{scalarC}. For this purpose, we define  the mass matrix $M_{\tGa} \in \RR^{n \times n}$  as
\begin{equation}\label{rigidity}
	M_{\tGa} = \sum_{i=1}^n |\Delta^+(\tGa)_i| M^i\,
	\qwhereq
	M^i_{h,j}=\int_{i/n}^{(i+1)/n}\xi_h\xi_j\,.
\end{equation} 
The elements of the matrices $M^i \in \RR^{n \times n}$ for $i=1,...,n$ are equal to zero excepted for the following block:  
\eq{
\begin{pmatrix}

M^i_{i,i}& M^i_{i,i+1}\\
M^i_{i+1,i}& M^i_{i+1,i+1}\\
\end{pmatrix}
=
\frac{1}{6n}
\begin{pmatrix}
2&1\\
1&2\\

\end{pmatrix}, 
}
where the indices $i-1$ and $i+1$ should be understood modulo $n$. 
 
 
\begin{prop}\label{prod-scla-cont} For all $\tPsi, \tPhi$ in $\CC^n$, one has
\begin{equation}\label{discreteprod}
 	\dotp{\tPhi}{\tPsi}_{\ell^2(\tilde{\GA})}
	= 
	\dotp{\tPhi}{M_{\tGa}\tPsi}_{\mathbb{C}^n}, 
\end{equation}
where $M_{\tGa}$ is the mass matrix defined in~\eqref{rigidity}.
\end{prop}

\begin{proof}  Denoting $\Phi=P_1(\tPhi)$ and $\Psi=P_1(\tPsi)$,
\eqref{representation-phi} and~\eqref{arc-length} imply that
\eq{
	\dotp{\Phi}{\Psi}_{L^2(\GA)}=\int_0^1 \,\Phi\cdot \Psi d\,\GA(s) 
	= 
	\sum_{i=1}^n |\Delta^+(\tGa)_i|\int_{i/n}^{(i+1)/n} 
	\left( \,\sum_{j=1}^n \tPhi_j \xi_j \cdot\sum_{h=1}^n \tPsi_h\xi_h\right) \d s\,.
}
Then, since \eqref{scal-prod-coord}, we have 
\begin{equation}
	\dotp{\tPhi}{\tPsi}_{\ell^2(\tilde{\GA})} 
	= 
	\sum_{i=1}^n |\Delta^+( \tGa)_i| \dotp{\tPhi}{M^i\tPsi}_{\mathbb{C}^n}
	= 
	\dotp{\tPhi}{M_{\tGa} \tPsi}_{\mathbb{C}^n}
\end{equation}
where $M_{\tGa}$ is the  mass matrix~\eqref{rigidity}.
\end{proof}

The next proposition  details the relationship between the product~\eqref{scal-prod-coord-sob} and the canonical inner product on $\CC^n$. To this end, we introduce  the  matrix $N_{\tGa} \in \RR^{n \times n}$  defined by
\begin{equation}\label{rigidity-2}
	N_{\tGa} = \sum_{i=1}^n |\Delta^+(\tGa)_i| N^i\,
	\qwhereq
	N^i_{h,j}=\frac{1}{|\Delta^+(\tGa)_j||\Delta^+(\tGa)_h|}\int_{i/n}^{(i+1)/n}\frac{\d\xi_h}{\d s}\cdot\frac{\d\xi_j}{\d s}\,.
\end{equation} 
The elements of the matrices $N^i \in \RR^{n \times n}$ for $i=1,...,n$ are equal to zero except for the following block:  
\eq{
\begin{pmatrix}

N^i_{i,i}& N^i_{i,i+1}\\
N^i_{i+1,i}& N^i_{i+1,i+1}\\
\end{pmatrix}
=
n
\begin{pmatrix}
\displaystyle{\frac{1}{|\Delta^+(\tGa)_i|^2}}&\displaystyle{-\frac{1}{|\Delta^+(\tGa)_i||\Delta^+(\tGa)_{i+1}|}}\\
\displaystyle{-\frac{1}{|\Delta^+(\tGa)_i||\Delta^+(\tGa)_{i+1}|}}&\displaystyle{\frac{1}{|\Delta^+(\tGa)_{i+1}|^2}}\\

\end{pmatrix}\,.
}
%In the following, we denote by $Q_{\tGa}$ the following matrix 
 
%\begin{equation}\label{matrix-Q}
%Q_{\tGa}=\,\mbox{diag}\left(\frac{1}{|\Delta^+(\tGa)_1|},\,...\,, \frac{1}{|\Delta^+(\tGa)_n|}\right)
%\end{equation}


\begin{prop}\label{prod-scla-cont-sob} For all $\tPsi, \tPhi$ in $\CC^n$, one has
\begin{equation}\label{discreteprod-sob}
 	\dotp{\tPhi}{\tPsi}_{h^1(\tilde{\GA})}
	= 
	\dotp{\tPhi}{U_{\tGa}\tPsi}_{\mathbb{C}^n}, 
\end{equation}
where $U_{\tGa}$ is the matrix defined by 
\begin{equation}\label{rigidity-mass-h1}
U_{\tGa} = M_{\tGa} +  N_{\tGa} \,,
\end{equation}
where $M_{\tGa}$, $N_{\tGa}$  are the   matrix \eqref{rigidity} and \eqref{rigidity-2} respectively. 
We point out that, since $U_{\tilde\Gamma}$ is a  matrix of an inner product in a basis,  it is always invertible.
\end{prop}

\begin{proof}  Denoting $\Phi=P_1(\tPhi)$ and $\Psi=P_1(\tPsi)$, \eqref{representation-phi} implies 
\eq{
	\langle \frac{\d \Phi}{\d \GA}, \frac{\d \Psi}{\d \GA}\rangle_{L^2(\GA)}=
	%\int_0^1 \,\frac{\d \Phi}{\d s}\cdot \frac{\d \Psi}{\d s}d\,\GA(s) = 
	\sum_{i=1}^n |\Delta^+(\tGa)_i|\int_{i/n}^{(i+1)/n} 
	\left( \,\sum_{j=1}^n \frac{\tPhi_j}{|\Delta^+(\tGa)_j|} \frac{\d \zeta_j}{\d s} \cdot\sum_{h=1}^n \frac{\tPsi_h}{|\Delta^+(\tGa)_h|}\frac{\d \zeta_h}{\d s}\right) \d s\,.
}
Then, by previous proposition, we have 
$$
	\dotp{\tPhi}{\tPsi}_{h^1(\tilde{\GA})} 
	= \dotp{\tPhi}{M_{\tGa} \tPsi}_{\mathbb{C}^n}	 
	+ \dotp{\tPhi}{N_{\tGa}  \tPsi}_{\mathbb{C}^n}
$$
where $M_{\tGa}$, $N_{\tGa}$  are the   matrices \eqref{rigidity} and \eqref{rigidity-2} respectively. 
%As $Q_{\tGa}$ is symmetric we get $$	\dotp{\tPhi}{\tPsi}_{h^1(\tilde{\GA})} 	= \dotp{\tPhi}{M_{\tGa} \tPsi}_{\mathbb{C}^n}	 	+ \dotp{ \tPhi}{Q_{\tGa}N_{\tGa} Q_{\tGa} \tPsi}_{\mathbb{C}^n}$$which proves the result.
 
\end{proof}

%%%%%%%%%%%%%%%%%%%%%%%%%%%%%%%%%%%%%%%%%%%%%%%%%%%%%%%%%%%%%%%%%%%%%%%%%%%%%%%%%%%%%%%
%%%%%%%%%%%%%%%%%%%%%%%%%%%%%%%%%%%%%%%%%%%%%%%%%%%%%%%%%%%%%%%%%%%%%%%%%%%%%%%%%%%%%%%
%%%%%%%%%%%%%%%%%%%%%%%%%%%%%%%%%%%%%%%%%%%%%%%%%%%%%%%%%%%%%%%%%%%%%%%%%%%%%%%%%%%%%%%
\subsection{Discrete Finsler Flow}

The initial optimization~\eqref{initial-eq} is discretized by restricting the minimization to the space $\Bb_n$, which corresponds to the following finite dimensional optimization
\eql{\label{eq-min-discr}
	\umin{\tGa \in \CC^n} \tE(\tGa)~,
}
where $\tilde E(\tGa)$ approximates $E(P_1(\tGa))$.

The discrete Finsler gradient is obtained in a similar way by restricting the optimization~\eqref{defgrad} to $\Bb_n$
\eql{\label{eq-finsler-grad-discr}
	\nabla_{\tR_{\tGa}} \tE(\tGa) \in 
	\uargmin{\tPhi \in \tilde\Ll_{\tGa}} \tR_{\tGa}(\tPhi)~, 
}
where the discrete penalty  reads
\eql{\label{defL-discrete}
	\tR_{\tGa}(\tPhi) = R_{P_1(\tGa)}(P_1(\tPhi))
%	\qandq
%	\tilde\Ll_{\tGa} = \enscond{\tilde\Phi \in \CC^n}{ P_1(\tPhi) \in \Ll_\Ga }
}
and, as discrete constraint, we set 
\eql{
	\label{eq-expression-Ll}
	\tilde{\Ll}_{\tGa} = \enscond{ \tPhi \in \CC^n }{
				\normbig{\tPi_{\tGa}( \nabla_{h^1(\tilde{\GA})} \tE(\tilde{\ga}) -\tPhi) }_{h^1(\tilde{\ga})} 
				\leq 
				\rho \normbig{\tPi_{\tGa}( \nabla_{h^1(\tGa)} \tE(\tilde{\ga})) }_{h^1(\tilde{\ga})} 
			} 
\,.}
The Finsler flow discretizing the original one~\eqref{subsec-finsler-descent} reads
\eql{\label{eq-finsler-flow-discr}
	\tGa_{k+1} = \tGa_k - \tau_k \nabla_{\tR_{\tGa_k}} \tE(\tGa_k).
}
where $\tau_k>0$ is chosen following the Wolfe rule~\eqref{Wolfe}.

The following sections detail how to compute this flow for the particular case of the curve matching energy introduced in Section~\ref{CM}. 

%%%%%%%%%%%%%%%%%%%%%%%%%%%%%%%%%%%%%%%%%%%%%%%%%%%%%%%%%%%%%%%%%%%%%%%%%%%%%%%%%%%%%%%
%%%%%%%%%%%%%%%%%%%%%%%%%%%%%%%%%%%%%%%%%%%%%%%%%%%%%%%%%%%%%%%%%%%%%%%%%%%%%%%%%%%%%%%
%%%%%%%%%%%%%%%%%%%%%%%%%%%%%%%%%%%%%%%%%%%%%%%%%%%%%%%%%%%%%%%%%%%%%%%%%%%%%%%%%%%%%%%
\subsection{Discrete Energy} 

%%%
\paragraph{Exact energy for piecewise affine curves. }

For curves $\GA=P_1(\tGa)$ and $\La=P_1(\tLa)$ in $\Bb_n$, the energy $E(\GA)$ defined in~\eqref{energy} can be computed as
\eq{
	E(\GA) = 
		\frac{1}{2}\Zz(\GA,\GA) - 
		\Zz(\Ga,\La) + 
		\frac{1}{2}\Zz(\La,\La)
}
where
\eq{
	\Zz(\Ga, \La) =
 	\sum_{i=1}^n \sum_{j=1}^n \dotp{ \Delta^+(\tGa)_i }{ \Delta^+( \tLa)_j } 
	T(\tGa,\tLa)_{i,j}
}
\eq{
	\qwhereq
	T(\tGa,\tLa)_{i,j} = \int_{\frac{i-1}{n}}^{ \frac{i}{n}} \int_{\frac{j-1}{n}}^{\frac{j}{n}} k(\GA(s), \La(t))\, \d \GA(s) \d \La(t)\,.
}

%%%
\paragraph{Approximate energy for piecewise affine curves. }

In general there is no closed form expression for the operator $T$, so that, to enable a direct computation of the energy and its gradient, we use a first order approximation with a trapezoidal quadrature formula
\eq{
	\tilde T(\tGa,\tLa)_{i,j} = 
	\frac{1}{4} \big(
		k(\tilde\GA_i,\tLa_j) + k(\tilde\GA_{i+1},\tLa_j) + 
		k(\tilde\GA_i,\tLa_{j+1}) + k(\tilde\GA_{i+1},\tLa_{j+1})
	\big)\,.
}
One thus has the approximation
\eq{
	\tilde T(\tGa,\tLa)_{i,j} =  T(\tGa,\tLa)_{i,j} + O(1/n^2). 
}
This defines the discrete energy $\tE$ on $\CC^n$ as 
\eql{\label{eq-energy-matching-discr}
	\tE(\tGa) = 
	\frac{1}{2}\tilde{\Zz}(\tGa,\tGa) - 
	\tilde{\Zz}(\tGa,\tLa) + 
	\frac{1}{2}\tilde{\Zz}(\tLa,\tLa)
}
\eq{
	\qwhereq
	\tilde{\Zz}(\tGa,\tLa) =
	\sum_{i=1}^n \sum_{j=1}^n \dotp{\Delta^+ (\tGa)_i }{ \Delta^+( \tLa)_j } \tilde T(\tGa,\tLa)_{i,j}
}


%%%
\paragraph{Discrete $h^1$-gradient. }

The following proposition gives the formula to calculate the gradient of $\tE$ with respect to inner product~\eqref{scal-prod-coord-sob}.
\begin{prop}
The gradient of $\tE$ at $\tGa$ with respect to the metric defined by the  inner product~\eqref{scal-prod-coord-sob} is
\eq{
	\nabla_{h^1(\tilde{\GA})} \tE(\tGa) = U_{\tGa}^{-1}\nabla \tE(\tGa)\,
}
where $U_{\tGa}$ is the  matrix~\eqref{rigidity-mass-h1} and $\nabla \tE$ the  gradient of $\tE$ for the canonical inner product of $\CC^{n}$~\eqref{scalarC}, which is given by

\begin{equation}\label{discretegradient}
\nabla \tE(\tGa)_i = \nabla \tilde{\Zz}(\tGa, \tGa)_i -\nabla \tilde{\Zz}(\tGa, \tLa)_i
\end{equation}
where 
$$
\begin{array}{ll}
 \nabla \tilde{\Zz}(\tGa, \tLa)_i &\displaystyle{ = 
 	\frac{1}{4} \sum_{j=1}^n  (\tGa_{i+1}-\tGa_{i-1}) (\tLa_{j+1}-\tLa_j)[\nabla_1 k(\tGa_i,\tLa_j)+ \nabla_1 k(\tGa_i,\tLa_{j+1})]} \\
		& \displaystyle{+ \sum_{j=1}^n  (\tLa_{j+1}-\tLa_j)[ T(\tGa,\tLa)_{i-1,j} - T(\tGa,\tLa)_{i,j} ]}\,.
		\end{array}
		$$

\end{prop}

\begin{proof}
The  gradient~\eqref{discretegradient} of $\tE$ for the canonical inner product of $\CC^{n}$  can be computed by a straightforward calculation. For every $\tPhi\in  \CC^n$ we have the following expression for the derivative of $\tE$

\eq{
	D \tE (\tGa)(\tPhi)= \dotp{\tPhi}{\nabla_{h^1(\tilde{\GA})} \tE(\tGa)}_{h^1(\tilde{\GA})}
	= 
	\dotp{\tPhi}{\nabla\tE(\tGa)}_{\mathbb{C}^n}
}
and, by~\eqref{discreteprod}, we get 
\eq{
	\nabla_{h^1(\tilde{\GA})} \tE(\tGa) = U_{\tGa}^{-1}\nabla \tE(\tGa)\,.
}
\end{proof}


%%%%%%%%%%%%%%%%%%%%%%%%%%%%%%%%%%%%%%%%%%%%%%%%%%%%%%%%%%%%%%%%%%%%%%%%%%%%%%%%%%%%%%%
%%%%%%%%%%%%%%%%%%%%%%%%%%%%%%%%%%%%%%%%%%%%%%%%%%%%%%%%%%%%%%%%%%%%%%%%%%%%%%%%%%%%%%%
%%%%%%%%%%%%%%%%%%%%%%%%%%%%%%%%%%%%%%%%%%%%%%%%%%%%%%%%%%%%%%%%%%%%%%%%%%%%%%%%%%%%%%%
\subsection{Discrete Piecewise-rigid Curve Matching}

This section first describes in a general setting the discrete Finsler gradient over finite-element spaces, then specializes it to the piecewise rigid penalty for the matching problem, and lastly gives the explicit formula of the corresponding functionals to be minimized numerically. 

%%%
\paragraph{Discrete piecewise-rigid penalty.}

To re-write conveniently the discrete Finsler gradient optimization~\eqref{eq-finsler-grad-discr}, we introduce the following finite-dimensional operators.

\begin{defn}[\BoxTitle{Discrete operators}]\label{dfn-discr-op} For all $\Ga=P_1(\tGa), \Phi=P_1(\tPhi)$ we define
	\begin{align*}
		\tV_{\tGa}(\tPhi) &= TV_{\GA}\left(\frac{\d \Phi}{\d \GA}\cdot {\bf n}_{\Ga} \right), \qquad \tV_{\tGa}: \CC^n\rightarrow \RR   \\
		% \tilde\Cc_{\tilde \Ga} &= \enscond{\tilde\Phi \in \CC^n}{ P_1(\tPhi) \in \Cc_\Ga } \\
		\tL_{\tGa}(\tPhi) &=P_0^{-1}(L_{\GA}^+(\Phi)),  
			\qquad \tL_{\tGa}: \CC^n\rightarrow \RR^n \\
		\tPi_{\tGa}(\tPhi) &=P_0^{-1}(\Pi_{\GA}(\Phi)), 
			\qquad \tPi_{\tGa}: \CC^n\rightarrow \CC^n
	\end{align*}
\end{defn}

The following proposition uses these discrete operators to compute the discrete Finsler penalty and constraint defined in~\eqref{defL-discrete}. 

\begin{prop}
We set $\tR_{\tGa}(\tPhi)= R_{P_1(\Ga)}(P_1(\Phi))$. One has 
\eql{\label{eq-discr-pr-penalty}
	\tR_{\tGa}(\tPhi) = \tV_{\tGa}(\tPhi) + \iota_{\tilde\Cc_{\tGa}}(\tPhi)
	\qwhereq
	\tilde\Cc_{\tGa} = \enscond{ \tPhi \in \CC^n }{ \tL_{\tGa}(\tPhi)=0 }.
}
\end{prop}
\begin{proof} 
Denoting $\Ga=P_1(\tGa), \Phi=P_1(\tPhi)$, by~\eqref{defL-discrete}, we have	
$$\tR_{\tGa}(\tPhi) = R_{\Ga}(\Phi)= TV_{\GA}\left(\frac{\d \Phi}{\d \GA}\cdot {\bf n}_{\Ga} \right)  + \iota_{\Cc_{\Ga}}(\Phi) = \tV_{\tGa}(\tPhi)  + \iota_{\tilde\Cc_{\tGa}}(\tPhi)$$
where 
$$\tilde\Cc_{\tGa}=\enscond{ \tPhi \in \CC^n }{ \tL_{\tGa}(\tPhi)=0 }.$$
%Analogously, by the definition of the inner product~\eqref{discreteprod}, we get \textcolor{red}{quel est le lien entre le gradient de $\tE$ et le gradient de $E$ ? Il faudrait $\tilde{ \nabla_{H^1} E(\ga)} = \nabla_{h^1(\tilde{\GA})} \tE(\tilde{\ga}) $ ..}
%$$
%\begin{array}{lll}
%\tilde{\Ll}_{\tGa}& =& \enscond{ \tPhi \in \CC^n }{
%				\normbig{\Pi_{\Ga}( \nabla_{H^1} \tE(\ga) -\Phi) }_{H^1({\ga)} }
%				\leq 
	%			\rho \normbig{\Pi_{\Ga}( %\nabla_{H^1} \tE(\ga)) }_{H^1(\ga)}}\\
%				&=& \enscond{ \tPhi \in \CC^n }{
%				\normbig{\tPi_{\tGa}( \nabla_{h^1(\tilde{\GA})} \tE(\tilde{\ga}) -\tPhi) }_{h^1(\tilde{\GA})(\tilde{\ga})} 
%				\leq 
%				\rho \normbig{\tPi_{\tGa}( \nabla_{h^1(\tilde{\GA})} \tE(\tilde{\ga})) }_{h^1(\tilde{\GA})(\tilde{\ga})}} .
%\end{array}				
%$$

\end{proof}

The following proposition gives explicit formulae for the discrete operators introduced in Definition~\ref{dfn-discr-op}.


\begin{prop}\label{constraints-discrete} For every $\tGa, \tPhi\in\CC^n$, we consider $\GA=P_1(\tGa)\in\Bb_n$, $\Phi=P_1(\tPhi)\in T_\GA \Bb_n$. One has 
\begin{align}\label{eq-expression-tL}
	\tL_{\tGa}(\tPhi)_i &= \dotp{\frac{\Delta^+(\tPhi)_i}{|\Delta^+(\tGa)_i|} }{ \frac{\Delta^+ (\tGa)_i}{|\Delta^+ (\tGa)_i|} }\, , \\
	%%
	\label{eq-expression-tPi}
	\tPi_{\tGa}(\tPhi)(s) &= \sum_{i=1}^n \langle \tPhi_i\xi_i(s) + \tPhi_{i+1} \xi_{i
+1}(s),  (\tilde{\bf n}_\ga)_i \rangle(\tilde{\bf n}_\ga)_i \zeta_i(s) \, ,
	  \\
	%%
	\label{eq-expression-tV}
		\tV_{\tGa}(\tPhi) & 
		% = \sum_{i=1}^n 
		% \left|
		%		\Delta^-\left(P_0^{-1}\left(  \frac{\d \Phi}{\d \GA}\cdot \ngam \right)\right)_i
		% \right|\, , \\
		% & 
		= \norm{\Delta^- ( \tilde H_{\tGa}(\tPhi) ) }_{\ell^1} 
		= 
		\sum_{i=1}^n
		\left| 
			\tilde H_{\tGa}(\tPhi)_i
			-
			\tilde H_{\tGa}(\tPhi)_{i-1}
		\right|\, , \\
	%%
	\mbox{where}\;\;\tilde H_{\tGa}(\tPhi)_i & := \dotp{ \frac{\Delta^+(\tPhi)_i}{|\Delta^+(\tGa)_i|} }{ (\tilde{\bf n}_\ga)_i }
\end{align}
and where $\tilde{\bf n}_\ga$ denotes the vector of the coordinates of ${\bf n}_\ga$ with respect to the basis of $\mathbb{P}_0$.
% and  $N_{\tGa} \in \RR^{N \times N}$ is the matrix defined as 
%\eql{\label{rigidity2}
%	N_{\tGa} = \sum_{i=1}^n |\Delta^+(\tGa)_i| N^i
%	\qwhereq
	%N^i_{h,j}=\int_{i/n}^{(i+1)/n}\xi_h\zeta_j\,.
%}
\end{prop}

\begin{proof} 
%%% 1 %%%%
\textbf{(Proof of~\eqref{eq-expression-tL})} Using~\eqref{first-derivative-approx}  the first derivative of $\Phi$ can be written (with respect to the basis of $\PP_{0,n}$) as 
\eq{
	\frac{\d \Phi}{\d \GA} = \sum_{i=1}^n \frac{\Delta^+(\tPhi)_i}{|\Delta^+(\tGa)_i|} \zeta_i\,
}
which implies that
\eq{
	L_{\GA}^+(\Phi)= \frac{\d \Phi}{\d \GA}\cdot \tgam = \sum_{i=1}^n \langle\frac{\Delta^+(\tPhi)_i}{|\Delta^+(\tGa)_i|} ,\frac{\Delta^+ (\tGa)_i}{|\Delta^+ (\tGa)_i|} \rangle\zeta_i\,.
}
Then, by the definitions of $L_{\GA}^{+(-)}$,  conditions $L_{\Ga}^{+(-)}(\Phi)=0$ become
\eq{
	\langle\frac{\Delta^+(\tPhi)_i}{|\Delta^+(\tGa)_i|} ,\frac{\Delta^+ (\tGa)_i}{|\Delta^+ (\tGa)_i|} \rangle=0 
	\quad \quad \foralls i=1,...,n,
}
which is equivalent to $\tL_{\tGa}(\tPhi)=0$.\\
%%% 2 %%%%
\textbf{(Proof of~\eqref{eq-expression-tPi})} By~\eqref{representation-phi} and~\eqref{tan-norm}, we get 
\eq{
\begin{array}{ll}
\Pi_\GA(\Phi)(s) & = \displaystyle{\langle \sum_{i=1}^n \tPhi_i\xi_i(s), \sum_{i=1}^n (\tilde{\bf n}_\ga)_i  \zeta_i(s) \rangle  \sum_{i=1}^n (\tilde{\bf n}_\ga)_i  \zeta_i(s)\,}\\
& =\displaystyle{  \sum_{i=1}^n \langle \tPhi_i \xi_i(s) + \tPhi_{i+1} \xi_{i+1}(s) ,  (\tilde{\bf n}_\ga)_i \rangle(\tilde{\bf n}_\ga)_i \zeta_i(s)}
\end{array}
}
which proves the result. \\
%%% 3 %%%%
\textbf{(Proof of~\eqref{eq-expression-tV})} By~\eqref{first-derivative-approx} and~\eqref{tan-norm}, we get
\begin{align*}
  TV_\GA\left( \frac{\d \Phi}{\d \GA}\cdot {\bf n}_\GA\right)
  &=  
  TV_\GA\left( \sum_{i=1}^n \langle\frac{\Delta^+(\tPhi)_i}{|\Delta^+(\tGa)_i|},(\tilde{\bf n}_\ga)_i \rangle\zeta_i \right)\\
  &= \sum_{i=1}^n
		\left| 
			\tilde H_{\tGa}(\tPhi)_i
			-
			\tilde H_{\tGa}(\tPhi)_{i-1}
		\right|
\end{align*}
where we used the fact that the  total variation for piecewise constant functions coincides with the sum of jumps sizes.
\end{proof}



%%%%%%%%%%%%%%%%%%%%%%%%%%%%%%%%%%%%%%%%%%%%%%%%%%%%%%%%%%%%%%%%%%%%%%%%%%%%%%%%%%%%%%%
%%%%%%%%%%%%%%%%%%%%%%%%%%%%%%%%%%%%%%%%%%%%%%%%%%%%%%%%%%%%%%%%%%%%%%%%%%%%%%%%%%%%%%%
%%%%%%%%%%%%%%%%%%%%%%%%%%%%%%%%%%%%%%%%%%%%%%%%%%%%%%%%%%%%%%%%%%%%%%%%%%%%%%%%%%%%%%%
\subsection{Calculation of the Discrete Finsler Gradient}
\label{calculation-finsler}

One can minimize the matching energy $\tE$ defined in~\eqref{eq-energy-matching-discr} using the Finsler flow $\{\tGa_k\}$ of~\eqref{eq-finsler-flow-discr}. This requires computing at each step $k$ the Finsler gradient~\eqref{eq-finsler-grad-discr} for the piecewise-rigid penalty $\tR_{\tGa}$ defined in~\eqref{eq-discr-pr-penalty}. Solving~\eqref{eq-finsler-grad-discr} at each step in turn requires the resolution of a finite dimensional convex problem, and the functional to be minimized is explicitly given with closed form formula in Proposition~\ref{constraints-discrete}.

Several convex optimization algorithms can be used to solve~\eqref{eq-finsler-grad-discr}. A convenient method consists in recasting the problem into a second order cone program by introducing additional auxiliary variables $(\tilde{\Phi}, \tilde{S},\tilde{Y},\tilde{T})$ as follow
\eq{
	\underset{ (\tilde{\Phi}, \tilde{S},\tilde{Y},\tilde{T}) \in \CC^{2n}\times\RR^{2n} }{ {\rm Min} }
		\dotp{\tilde{Y} }{ {\bf 1}}_{\mathbb{C}^n} 	
		\qwhereq {\bf 1}=(1,...,1) \in \RR^n
}
where the minimum is taken under the following set of affine and conic constraints
\begin{align*}
	& -\tilde{Y}_i\leq \Delta^-( \tilde{H}_{\tilde \Gamma} (\tilde \Phi) )_i  \leq \tilde{Y}_i \;,\quad \quad  \foralls i=1,...,n \\
	& \tL_{\tGa}(\tilde{\Phi}) =0 \\
	& \tilde{S} =  U_{\tGa}^{1/2}(\tPi_{\tilde{\Ga}}(\tilde{\Phi}) - \tPi_{\tilde{\Ga}}(\nabla_{h^1(\tilde{\GA})} \tE(\tilde{\Ga}))) \\
	& \langle \tilde{T}, {\bf 1}\rangle_{\mathbb{C}^n} \leq \rho^2 \normbig{\tPi_{\tGa}( \nabla_{h^1(\tilde{\GA})} \tE(\tilde{\ga})) }_{h^1(\tilde{\ga})}^2  \\
	& (\tilde{S}_i,\tilde{T}_i) \in 
		\enscond{ (s,t) \in \CC \times \RR }{ |s|^2 \leq t } ,\quad \foralls i=1,...,n.
\end{align*}
We point out that the variable $\tilde{S}$ is defined by the mass matrix $U_{\tGa}$ \eqref{rigidity-mass-h1} because of the relationship \eqref{discreteprod-sob}. For the numerical simulation, we use an interior point solver, see~\cite{convex-optimization}. These interior points algorithms are powerful methods to solve medium scale SOCP problems and work remarkably well for $n$ up to several thousands, which is typically the case for the curve matching problem.
