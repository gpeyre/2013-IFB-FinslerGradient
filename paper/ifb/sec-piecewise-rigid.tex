
%%%%%%%%%%%%%%%%%%%%%%%%%%%%%%%%%%%%%%%%%%%%%%%%%%%%%%%%%%%%%%%%%%%%%%%%%
%%%%%%%%%%%%%%%%%%%%%%%%%%%%%%%%%%%%%%%%%%%%%%%%%%%%%%%%%%%%%%%%%%%%%%%%%
%%%%%%%%%%%%%%%%%%%%%%%%%%%%%%%%%%%%%%%%%%%%%%%%%%%%%%%%%%%%%%%%%%%%%%%%%	
\section{Piecewise Rigidity}\label{PWR}

This section defines a penalty $R_\GA$ that favors piecewise rigid $BV^2$ deformations of $\BVD$-curves.  For the sake of clarity we present the construction of this penalty in two steps. We first characterize in Section~\ref{GRD} $C^2$-global rigid deformations for  smooth curves. Then, in Section~\ref{BV2M}, we introduce a  penalty that favors  piecewise rigid $BV^2$ deformations for curves belonging to $\Bb$.


%%%%%%%%%%%%%%%%%%%%%%%%%%%%%%%%%%%%%%%%%%%%%%%%%%%%%%%%%%%%%%%%%%%%%%%%%
\subsection{Rigid Curve Deformations}\label{GRD}

A smooth curve evolution $\GA_t\in C^1(\RR, \Bb)$ reads
\eql{\label{eq-continuous-flow}
	\foralls t \in \RR, \quad \pd{\GA_t(s)}{t} = \Phi_t(s)
	\qwhereq
	\Phi_t \in T_{\GA(t)} \Bb\,.
}
 We further assume in this section that $\GA_t$ is a $C^2$ curve. This evolution is said to be globally rigid if it preserves the pairwise distances between points along the curves, i.e. 
\eql{\label{eq-rigid-flow}
	\foralls t \in \RR,\quad \foralls (s,s') \in \Circ \times \Circ, \quad
	|\GA_t(s)-\GA_t(s')|=|\GA_0(s)-\GA_0(s')|.
} 
The following proposition shows that the set of instantaneous motions $\Phi_t$ giving rise to a rigid evolution  is, at each time, a linear sub-space of dimension 3 of $T_{\GA_t} \Bb$.

\begin{prop}\label{rigid-ch}
	The evolution~\eqref{eq-continuous-flow} satisfies~\eqref{eq-rigid-flow} if and only if $\Phi_t \in \Rr_{\GA_t}$ for all $t \in \RR$,  where  
	\eql{\label{eq-rigid-vector-space}
		\Rr_\GA = \enscond{ \Phi \in T_\Ga \Bb }{
                        \exists a \in \RR, \exists b \in \R^2,\;\;
			\foralls s \in \Circ, \; \Phi(s) =  a \, \rot{\GA(s)}  + b
		}\,.
	}
\end{prop}
\begin{proof}
Recall that the group of distance preserving transformations on $\R^d$ is the Euclidean group $E(d) = \R^d \rtimes  O_d(\R)$ and that any element of $E(d)$ is uniquely defined by the image of $d+1$ points in $\R^d$ which are affinely independent. Therefore, provided that the curve $\GA$ has at least three non-collinear points, $\Phi_t$ is the restriction of $g_t \in E(d)$, a path on $E(d)$ which is uniquely defined. In addition, $g_t$ and $\Phi_t$ have the same smoothness. Thus the result follows from the fact that the Lie algebra of $\R^d \rtimes  O_d(\R)$ is $\R^d \rtimes A(d)$, where $A(d)$ denotes the set of antisymmetric matrices.
The degenerate cases such as when the curve is contained in a line or a point are similar.
%By the definition of evolution, we have\begin{equation}\label{second-rigid00}\GA_t(s)= \GA_0(s)+ \int_0^t \Phi_{\tau}(s)\d \tau \quad \forall s\in \Circ\,.\end{equation} If the evolution satisfies~\eqref{eq-rigid-flow}, then the application $m(t):\GA_0(\Circ)\rightarrow \GA_t(\Circ)$ defined as$$m(t)(\GA_0(s))=\GA_0(s)+ \int_0^t \Phi_{\tau}(s)\d \tau \quad \forall s\in \Circ\,$$ is an isometry between the supports of the curves $\GA_0$ and $\GA_t$. Now, by considering three non-collinear points of $\GA_0(\Circ)$ denoted by $\GA_0(s_i)$ ($i=1,2,3$), we can extend such an isometry $m(t)$ to an isometry of the plane. We define     $\tilde{m}(t): \RR^2 \rightarrow\RR^2$  as  follows: for every $x\in \RR^2$, $\tilde{m}(t)(x) $ is such that \begin{equation}\label{condition-isometry} |\tilde{m}(t)(x) -\GA_t(s_i)|=|x-\GA_0(s_i)|\quad i=1,2,3\,. \end{equation} The previous condition determines a unique isometry of the plane $\tilde{m}(t)$ such that $$\tilde{m}(t)(\GA_0(s))=m(t)(\GA_0(s))\quad \forall s\in \Circ\,.$$\par Now, by the characterization of the isometries,  $\tilde{m}(t)$ can be written as the composition of a rotation and a translation, so that\begin{equation}\tilde{m}(t)(x) = A(t)x + c(t)\quad \forall x\in \RR^2\end{equation}for some  vector $c(t)$ and matrix $A(t)$ defined by$$\begin{pmatrix}\cos\theta(t)&-\sin\theta(t)\\\sin\theta(t)&\cos\theta(t)\\\end{pmatrix} \,. $$Note that $A(t)$ and $c(t)$ are defined by  condition~\eqref{condition-isometry} as two differentiable functions with respect to $t$. Moreover, as $\tilde{m}(t)$ is continuous with respect to $t$, it is a direct isometry,  which justifies the previous definition of $A(t)$.  In particular, as $\tilde{m}(t)$ coincides with $m(t)$ on $\GA_0(\Circ)$, we have\begin{equation}\label{first-rigid0}\GA_t(s)= A(t)\GA_0(s) + c(t)\quad \forall s\in \Circ\,.\end{equation}By differentiating~\eqref{second-rigid00} and~\eqref{first-rigid0}  with respect to $t$, we get\begin{equation}\label{third-rigid0}\Phi_t(s)= A'(t)\GA_0(s) + c'(t) \quad \forall s\in \Circ, \end{equation}where $(\cdot)'$ denotes here the derivative with respect to $t$. Note that, by the definition of a rotation matrix in dimension two, we have \begin{equation}\label{four-rigid}	A'(t)A(t)^{-1}w= \theta'(t){w}^\bot\end{equation}for every $\rm{w}\in \RR^2$. Then by~\eqref{first-rigid0},~\eqref{third-rigid0}, and~\eqref{four-rigid} we get, for all $s\in \Circ$, $$\Phi_t(s)= A'(t)A(t)^{-1}\GA_t(s) + c'(t)- A'(t)A(t)^{-1}c(t) =\theta'(t)\GA_t(s)^\bot + c'(t)- A'(t)A(t)^{-1}c(t), $$which implies that $\Phi_t \in \Rr_{\GA_t}$ with $a=\theta'(t)$ and $b=c'(t)- A'(t)A(t)^{-1}c(t)$.\par On the other hand if $\Phi_t \in \Rr_{\GA_t}$ then $$\Phi_t(s)-\Phi_t(s')=a(\GA_t(s)-\GA_t(s'))^\bot \quad \forall(s,s')\in \Circ\times\Circ\,.$$Then$$\frac{\partial}{\partial t}\norm{\GA_t(s)-\GA_t(s')}^2=2(\Phi_t(s)-\Phi_t(s'))\cdot(\GA_t(s)-\GA_t(s'))=0 \quad \forall(s,s')\in \Circ\times\Circ$$which implies~\eqref{eq-rigid-flow}. 
\end{proof}

Note that for numerical simulations, one replaces the continuous PDE~\eqref{eq-continuous-flow} by a flow discretized at times $t_k = k \tau$ for some step size $\tau>0$ and $k \in \NN$, 
\eq{
	\GA_{k+1} = \GA_k + \tau \Phi_k
	\qwhereq
	\Phi_k \in T_{\GA_k} \Bb.
} 
This is for instance the case of a gradient flow such as~\eqref{sequence} where $\Phi_k = - \nabla_{R_{\GA_k}} E(\GA_k)$. In this discretized setting, imposing $\Phi_k \in \Rr_{\GA_k}$ only guarantees that rigidity~\eqref{eq-rigid-flow} holds approximately and for small enough times $t_k$. 

The following proposition describes this set of tangent fields in an intrinsic manner (using only derivatives along the curve $\GA$), and is pivotal to extend $\Rr_\GA$ to piecewise-rigid tangent fields.

\begin{prop}\label{charactrigid}
	For a $C^2$-curve $\GA$, one has $\Phi \in \Rr_\GA$ if and only if $\Phi$ is $C^2$ and satisfies $L_\GA(\Phi) = 0$ and $H_\GA(\Phi) = 0$, where $L_\GA$ and $H_\GA$ are the following linear operators
\begin{equation}\label{eqrigid}
 	L_\GA(\Phi) \;=\;   \frac{\d\Phi}{\dgs}\, \cdot\, \tgam  
	\qandq  
	H_\GA(\Phi) \;=\;  \frac{\d^2\Phi}{\dgs^2} \, \cdot\, \ngam \,.
\end{equation}
From a geometric point of view, $L_\GA(\Phi)$ takes into account the length changes and $H_\GA(\Phi)$ the curvature changes.
\end{prop}

\begin{proof}
Using the parameterization of $\Ga$, any such deformation $\Phi$ satisfies 
\begin{equation}
	\label{eq:rigidg}
	\exists\, !\; (a,b) \in \R\times\R^2, \quad \foralls s \in [0,1], \quad \Phi(s) \; = \;  a\rot{\Ga(s)}  + b \,.
\end{equation}
By differentiation with respect to \! $s$, this is equivalent to 
\eq{ 
	\exists\, !\; a \in \R,\quad \foralls s \in [0,1], \quad \frac{\d\Phi}{\d s}(s) \; = \; a|\GA'(s)|\; \ngam(s)
}
which can be rewritten as $\frac{\d\Phi}{\d\GA} (s) =  a \ngam(s) $ by differentiating with respect to \! the length element $\d\GA = \norg\,\d s$,
or simply as $ \frac{\d\Phi}{\d s}(s)  = a \ngam(s) $ 
by considering an arc-length parameterization. 
This is equivalent to 
\eq{
	\exists\, !\; a \in \R,\quad \foralls s \in [0,1], \quad 
	\begin{cases} 
	\displaystyle \;\; \frac{\d\Phi}{\dgs}\, \cdot\, \tgam(s) \; = \; 0 \vspace{2mm} \\
	\displaystyle \;\; \frac{\d\Phi}{\dgs}\, \cdot\, \ngam(s) \; = \; a
	\end{cases}
}
which is  equivalent to
\eq{
	\begin{cases}
		\displaystyle \;\; \frac{\d\Phi}{\dgs}\, \cdot\, \tgam \; = \; 0 \vspace{2mm} \\
		\displaystyle \;\; \frac{\d}{\dgs} \left( \frac{\d\Phi}{\dgs} \, \cdot\, \ngam \right) \; =\; 0
	\end{cases}
}
and, using that 
\eq{ 
	\frac{\d}{\dgs} \left( \frac{\d\Phi}{\dgs} \, \cdot\, \ngam \right) \; =\;  
	\frac{\d^2\Phi}{\dgs^2} \, \cdot\, \ngam \; - \frac{\d\Phi}{\dgs} \, \cdot\, \kappa_\Ga \,\tgam , 
}
where $\kappa_\Ga$ is the curvature of $\GA$, we obtain the desired characterization.
\end{proof} 





%%%%%%%%%%%%%%%%%%%%%%%%%%%%%%%%%%%%%%%%%%%%%%%%%%%%%%%%%%%%%%%%%%%%
\subsection{Piecewise rigid $BV^2$ deformations }\label{BV2M}

This section extends the globally rigid evolution considered in the previous section to piecewise-rigid evolution.
\par In the smooth case considered in the previous section, this corresponds to imposing that an instantaneous deformation $\Phi \in T_\GA \Bb$  satisfies~\eqref{eqrigid} piecewisely for possibly different pairs $(a,b)$ on each piece.
To generate a piecewise-smooth Finsler gradient $\Phi = \nabla_{R_\GA} E(\GA)$ (as defined in~\eqref{defgrad}) that is a piecewise rigid deformation, one should design a penalty $R_\GA$ that satisfies this property. This is equivalent to imposing  
$L_\GA(\Phi) = 0$ and $H_\GA(\Phi) = 0$ for all $s \in [0,1]$ except for a finite number of points (the articulations between the pieces). In particular, note that $L_\GA(\Phi)$ is undefined at these points, while $H_\GA(\Phi)$ is the sum of Dirac measures concentrated at the  articulation points (due to the variations of $a$). This suggests that, in the smooth case, we can favor piecewise rigidity by minimizing $\left\| H_\GA(\Phi) \right\|_{L^1(\Ga)}$ under the constraint $L_\GA(\Phi) = 0 \;\; a.e.$, so that  we control the jumps of the second derivative without setting in advance the articulation points. Note also that the minimization of the $L^1$-norm favors sparsity and, in contrast to the $L^2$-norm, it enables the emergence of Dirac measures.
\par In order to extend such an idea to the $BV^2$-framework we remind that 
$$\left\| H_\GA(\Phi) \right\|_{L^1(\Ga)}=TV_\GA\left(\frac{\d \Phi}{\d\GA}\cdot \ngam\right)\quad \forall \,\Phi\in C^{2}(\Circ,\RR^2),\, \GA \in C^2(\Circ, \RR^2)$$
which defines a suitable penalty in the $BV^2$-setting.
Moreover, since we are interested in piecewise rigid motions, we  deal with curves that could be not  $C^1$ at some points $s$. It is useful to introduce the following operators
\begin{align}\label{eq-operator-L}
	L_\GA^+(\Phi)(s) &=	\underset{\underset{t\in(s,s+\epsilon)}{t\rightarrow s}}{\lim} 
		\frac{\d\Phi}{\dgs}(t) \cdot\, \tgam(t)\,, \\
	L_\GA^-(\Phi)(s) &=\underset{\underset{t\in(s-\epsilon,s)}{t\rightarrow s}}{\lim} 
		\frac{\d\Phi}{\dgs}(t) \cdot\, \tgam(t)\,.
\end{align}
Of course if $\GA$ and $\Phi$ are $C^1$ at $s$ we have $L_\GA^+(\Phi)(s)=L_\GA^-(\Phi)(s)=L_\GA(\Phi)(s)$. 
The next definition introduces a penalty for piecewise rigid evolution in $\Bb$.

\begin{defn}[\BoxTitle{$\BVD$ Piecewise-rigid penalty}]\label{defn1}
	For $\Ga \in \Bb$ and $\Phi \in T_\GA\Bb = \BVDga$, we define
	\begin{equation}\label{eq-piecewise-rigid-penalty}	
		R_\Ga(\Phi) =
			TV_\GA\left(\frac{\d \Phi}{\d\GA}\cdot \ngam\right)
			 + \iota_{\Cc_\Ga}(\Phi)
	\end{equation}
where $\iota_{\Cc_\Ga}$ is the indicator  function of $\Cc_\Ga$ 
$$
\displaystyle{\iota_{\Cc_\Ga}(\Phi)=
\left\{
\begin{array} {ll}
0 &\text{if } \Phi \in \Cc_\Ga\\
+\infty &\text{otherwise }  
\end{array}\right.}
\,.$$
Note that~\eqref{TV} is the total variation of $f$ with respect to the measure $\d\GA$. We remind  that $TV_\GA(f)=|Df|(\Circ)$ for every $f\in L^1(\Circ,\RR^2)$.
\par The set $\Cc_\Ga$ is defined as follows
\begin{equation}\label{eq-constr-C-Ga}
	\Cc_\Ga = \enscond{\Phi\in T_\GA\Bb }{L^+_\GA(\Phi)=L^-_\GA(\Phi)=0 }\,.
\end{equation}
\end{defn}
In order to define the Finsler gradient  we consider a constraint on the normal component of the deformation field.

\begin{defn}[\BoxTitle{Deviation constraint}]\label{defn-dev-constr} For $\GA \in \Bb$, we define
\eql{\label{eq-dev-constr}
	\Ll_\GA =\enscond{
		 \Phi\in T_\Ga\Bb
	}{ 
		\normbig{ \Pi_\GA( \nablad E(\Ga) -\Phi )}_{W^{1,2}(\Ga)} \leq \rho \normbig{\Pi_\GA(\nablad E(\Ga))  }_{W^{1,2}(\Ga)}
	}\,.
}
\end{defn}

Here, $\rho\in (0,1)$ is called the rigidification parameter, and controls the deviation of the Finsler gradient from the $W^{1,2}$ gradient. $\Pi_\GA$ is the projector introduced in equation~\eqref{normal-proj}.
\par We point out that in the applications studied in this paper we consider an intrinsic energy $E$ (i.e., it does not depend on reparameterization). In this case  the $W^{1,2}$-gradient of $E$ is normal to the curve, so that  $\Ll_\GA$ satisfies condition~\eqref{cond-constraints} in the case of an intrinsic energy, and  it can be used to define a valid Finsler gradient. 




 Using these specific instantiations for $R_\GA$ and $\Ll_\GA$, Definition~\ref{FinslerGrad} reads in this $\BVD$ framework as follows.

\begin{defn}[\BoxTitle{$\BVD$ Piecewise-rigid Finsler gradient}] We define
\begin{equation}\label{grad-rigid}
	\nabla_{R_\GA} E(\GA) \in \uargmin{\Phi \in \Ll_\GA}\;  
   R_\GA(\Phi).
\end{equation}
\end{defn}

The following result  ensures the existence of a Finsler gradient. To prove  it we consider the space $\Bb$ equipped with the weak* toplogy.

\begin{thm}\label{existence1} 
	Problem~\eqref{grad-rigid}  admits at least a solution.
	
\end{thm}

In order to prove the theorem, we need the following lemmas. They guarantee in particular the compactness of minimizing sequences with respect to the $BV^2$-weak* topology. The proof relies on the evaluation of a bilinear form which is degenerate if the curve is a circle, so we treat the case of the circle separately. 

\begin{lem}\label{lem-compacity}
Let  $\GA\in \mathcal{B}$ be an injective curve.  We suppose that $\GA$ is different from a circle. Then there exists a constant $C(\GA)$ depending on $\GA$ such that 
\begin{equation}\label{compactnessR-1}
	 	\norm{ \Phi }_{BV^{2}(\GA)} \leq 
		C(\GA)\left( (1+\rho) \|\Pi_\GA(\nabla_{W^{1,2}(\Ga)}E(\GA))\|_{W^{1,2}(\Ga)} + R_\GA(\Phi) \right) \quad \forall\,\Phi \in \Ll_\GA \cap \Cc_\GA
\end{equation}
where $\Pi_\GA$ is the operator defined in \eqref{normal-proj}.	
\end{lem}

\begin{proof} 
The proof is essentially based on   estimation~\eqref{bound-first-der} giving a bound for the $L^\infty$-norms of the deformation $\Phi$ and its first derivative. We also remark that, as $\Phi\in \Ll_\GA$, we have
\begin{equation}\label{l1-bound}
\norm{ \Pi_\GA(\Phi) }_{
L^{2}(\Ga)} \leq (1+\rho)\|\Pi_\GA(\nabla_{W^{1,2}(\Ga)}E(\GA))\|_{W^{1,2}(\Ga)}\,.
\end{equation}
In the following we denote by $l(\GA)$ the length of the curve $\GA$. 

{\bf Bound on the first derivative}. In this section  we prove the following   estimate for the $L^\infty$-norms of  $\frac{\d\Phi}{\d\GA}\cdot \ngam $ and $\Phi$:
\begin{equation}\label{bound-first-der}
	\mbox{max}\left\{
		\left\|\Phi\right\|_{L^{\infty}(\GA)}\,;\,
		\left\|\frac{\d \Phi}{\d\GA}\cdot\ngam\right\|_{L^{\infty}(\GA)}
	\right\}
	\leq C_0(\GA)\left((1+\rho)\|\Pi_\GA(\nabla_{W^{1,2}(\Ga)}E(\GA))\|_{W^{1,2}(\Ga)}+ R_\GA(\Phi)\right)\,
\end{equation}
where $C_0(\GA)$ depends on  $\GA$. 
\par  Let $s_0\in \Circ$, we can write
$$\frac{\d\Phi}{\d\GA}\cdot \ngam = u + a$$
where $u\in BV(\GA)$ such that $u(s_0)=0$  and $a=\frac{\d\Phi}{\d\GA}(s_0)\cdot \ngam(s_0)\in \R$. As $\Phi\in\Cc_\GA$ we have $L_\GA^+(\Phi)=L_\GA^-(\Phi)=0$, which implies
$$\frac{\d\Phi}{\d\GA}=\left(\frac{\d\Phi}{\d\GA}\cdot \ngam\right)\ngam$$
and
\begin{equation}\label{writing-phi}
\Phi(s)= \Phi(s_0) + a[\GA(s)-\GA(s_0)]^\bot+ \int_{s_0}^s u\ngam \d \GA(s) \,\quad \forall \, s\in \Circ\,.
\end{equation}
Now, by projecting on the normal to $\GA$, we can write 
\begin{equation}\label{decomp-xi}\Pi_\GA(\Phi)=\Pi_\GA(\Phi(s_0)+ a[\GA(s)-\GA(s_0)]^\bot)+\Pi_\GA\left(\int_{s_0}^s u\ngam \d \GA(s)\right)\,.
\end{equation}
In particular, by the properties of good representatives for  $BV$-functions of one variable (see~\cite{AFP} p. 19), we have 
$$
|u(s)|=|u(s)-u(s_0)|\leq TV_\GA(u)\quad \forall s\in \Circ
$$
which implies that
\begin{equation}\label{first-bv0}
\left\|\int_{s_0}^s u\ngam \d \GA(s)\right\|_{L^\infty(\GA)}\leq
l(\GA) TV_\GA(u)= l(\GA) R_\GA(\Phi)\,
 \end{equation}
 and
 \begin{equation}\label{first-bv0-l2}
\left\|\int_{s_0}^s u\ngam \d \GA(s)\right\|_{L^2(\GA)}\leq
l(\GA)^{3/2} R_\GA(\Phi)\,
 \end{equation}
Thus, by~\eqref{l1-bound},~\eqref{first-bv0-l2}, and~\eqref{decomp-xi} it follows that
\begin{equation}\label{bound-xi}
\|\Pi_\GA(\Phi(s_0)+ a[\GA(s)-\GA(s_0)]^\bot)\|_{L^2(\GA)}\leq (1+\rho)\|\Pi_\GA(\nabla_{W^{1,2}(\Ga)}E(\GA))\|_{W^{1,2}(\Ga)}+ l(\GA)^{3/2}R_\GA(\Phi)\,.
\end{equation}
\par We remark now that $\|\Pi_\GA(\Phi(s_0)+ a[\GA(s)-\GA(s_0)]^\bot)\|_{L^2(\GA)}^2$ can be written as   
\begin{equation}\label{bilinear}
\|\Pi_\GA(\Phi(s_0)+a[\GA(s)-\GA(s_0)]^\bot)\|_{L^2(\GA)}^2=(|\Phi(s_0)|, a)\cdot A\left(\frac{\Phi(s_0)}{|\Phi(s_0)|},s_0\right)
\begin{pmatrix}
|\Phi(s_0)|\\
a\\
\end{pmatrix}
\end{equation}
where, for any $e\in \Circ\subset \RR^2$ and $s_0\in \Circ$, the matrix $A(e,s_0)$ is defined by
\begin{equation}
\begin{pmatrix}
\int_{\Circ} \left(e\cdot\ngam\right)^2\d \GA(s)&\int_{\Circ}\left([\GA(s)-\GA(s_0)]^\bot\cdot\ngam\right)\left(e\cdot\ngam\right) \d \GA(s)\\
\int_{\Circ}\left([\GA(s)-\GA(s_0)]^\bot\cdot\ngam\right)\left(e\cdot\ngam\right) \d \GA(s)&\int_{\Circ} \left([\GA(s)-\GA(s_0)]^\bot\cdot\ngam\right)^2\d \GA(s)\\
\end{pmatrix}
.
\end{equation}

\par Note that the bilinear form defined by $A(e,s_0)$ is degenerate if and only if the determinant of $A(e,s_0)$ is zero which means that there exists  $\alpha\in\R$ such that 
$(e- \alpha[\GA(s)-\GA(s_0)]^\bot)\cdot \ngam=0$ for every $s\in \Circ$. Note that this  implies that $\GA$ is either a circle or a line. Now, as we work with closed injective curves $\GA$ is different from a line. Then, because of the hypothesis on $\GA$, we get that for every $s_0\in \Circ$ the   bilinear form associated with $A(e,s_0)$  is not degenerate.

\par In particular the determinant of $A$ is positive which means that the bilinear form  is positive-definite.  This implies  that its smallest eigenvalue is positive and in particular, by a straightforward calculation, it can be written as $\lambda\left(e,s_0\right)$ where   $\lambda:\Circ\rightarrow \R$ is a positive continuous function. Then, we have
\begin{equation}\label{case1}
\underset{e,s_0\in \Circ}{\inf}\,\lambda(e, s_0)(|\Phi(s_0)|^2+a^2) \leq \lambda\left(\frac{\Phi(s_0)}{|\Phi(s_0)|},s_0\right)(|\Phi(s_0)|^2+a^2)\leq \|\Pi_\GA(\Phi(s_0)+ a\GA(s)^\bot)\|_{L^2(\GA)}^2\,
\end{equation}
where the infimum of $\lambda$ on $\Circ\times\Circ$ is a positive constant depending only on $\GA$ and denoted by $\lambda_\GA$. 
\par The previous relationship and~\eqref{bound-xi} prove that, for every $s_0\in \Circ$, we have 
\begin{equation}\label{norme-infty}
	\mbox{max}\left\{|\Phi(s_0)|,a\right\}
	\leq C_0(\GA)\left((1+\rho)\|\Pi_\GA(\nabla_{W^{1,2}(\Ga)}E(\GA))\|_{W^{1,2}(\Ga)}+ R_\GA(\Phi)\right)\,
\end{equation}
where $C_0(\GA)=\mbox{max}\{1/\lambda_\GA, l(\GA)^{3/2}/\lambda_\GA\}$ depends only on $\GA$.

Then, because of the arbitrary choice of $s_0$ and the definition of $a$ ($a=\frac{\d\Phi}{\d\GA}(s_0)\cdot \ngam(s_0)$),~\eqref{norme-infty}      implies~\eqref{bound-first-der}. In particular~\eqref{bound-first-der}  gives a bound for the $W^{1,1}(\GA)$-norm of $\Phi$. 

{\bf Bound on the second variation}. 
We have
$$ TV^2_\GA(\Phi) = TV_\GA\left(\frac{\d\Phi}{\d\GA}\right)\,.$$ 
Now,  $\frac{\d \Phi}{\d\GA}=\left(\frac{\d \Phi}{\d\GA}\cdot \ngam\right)\ngam\in BV(\GA)$ and, by the generalization of the  product rule  to  $BV$-functions (see Theorem 3.96, Example 3.97, and Remark 3.98 in~\cite{AFP}), we get 
$$
TV_\GA\left(\frac{\d \Phi}{\d\GA}\right)\leq 2\left(TV_\GA\left(\frac{\d \Phi}{\d\GA}\cdot \ngam\right) + \left\|\frac{\d \Phi}{\d\GA}\cdot\ngam\right\|_{L^\infty(\GA)}TV_\GA(\ngam)\right) \,.
$$
The constant 2 in the previous inequality comes from the calculation of the total variation on the intersection of the jump sets of $\left(\frac{\d \Phi}{\d\GA}\cdot \ngam\right)$ and $\ngam$ (see Example 3.97 in~\cite{AFP}).
Note also that $TV_\GA(\ngam)=|D \ngam|(\Circ)$. 
\par Then, by~\eqref{deriv-norm} and~\eqref{curvature}, we get 
\begin{equation}\label{estimate-second-var0}
\begin{array}{ll}
\displaystyle{TV_\GA\left(\frac{\d \Phi}{\d\GA}\right)}&\displaystyle{\leq 2\left(TV_\GA\left(\frac{\d \Phi}{\d\GA}\cdot \ngam\right) + |\rm{curv}_\GA|(\Circ)\left\| \frac{\d \Phi}{\d\GA}\cdot\ngam\right\|_{L^\infty(\GA)}\right)}\\
&\displaystyle{\leq  2\left(TV_\GA\left(\frac{\d \Phi}{\d\GA}\cdot \ngam\right) + |D^2\GA|(\Circ)\left\| \frac{\d \Phi}{\d\GA}\cdot\ngam\right\|_{L^\infty(\GA)}\right)}\,\\
\end{array}
\end{equation}
which implies that 
\begin{equation}\label{estimate-second-var}
TV_\GA^2\left(\Phi\right)\leq C_1(\GA)\left(R_\GA(\Phi) + \left\|\frac{\d \Phi}{\d\GA}\cdot\ngam\right\|_{L^\infty(\GA)}\right)\,.\\
\end{equation}
where $C_1(\GA)$ is a constant depending on $\GA$.

The Lemma follows from~\eqref{bound-first-der} and~\eqref{estimate-second-var}.
\end{proof}

The next lemma gives a similar result in the case where $\GA$ is a circle.

\begin{lem}\label{lem-compacity-circle}
Let  $\GA\in \mathcal{B}$ be a circle with radius $r$.  Then there exists a constant $C(r)$ depending on $r$ such that 
	\begin{equation}\label{compactnessR}
	 	\norm{ \Phi }_{BV^2(\GA)} \leq 
		C(r)\left( (1+\rho) \|\Pi_\GA(\nabla_{W^{1,2}(\Ga)}E(\GA))\|_{W^{1,2}(\Ga)} +
R_\GA(\Phi)\right)\, 
	\end{equation}
	for every $\Phi \in \Ll_\GA \cap \Cc_\GA$ such that $\Phi(s_0)\cdot\tgam(s_0) =0$ for some $s_0\in\Circ$.
\end{lem}

\begin{proof}
 The proof is based on the same arguments used to prove the previous lemma. We denote by $r$ the radius of the circle.
\par As $\Phi(s_0)\cdot\tgam(s_0)=0$, 
by the properties of good representatives for  $BV$-functions of one variable (see~\cite{AFP} p. 19), we have 
\begin{equation}\label{estim00}
|\Phi\cdot\tgam|=|\Phi(s)\cdot\tgam(s)-\Phi(s_0)\cdot\tgam(s_0)|\leq TV_\GA(\Phi\cdot\tgam)\quad \forall s\in \Circ\,.
\end{equation}
Now, as $L_\GA^+(\Phi)=L_\GA^-(\Phi)=0$ and the curvature is equal to $1/r$ at each point, we get 
$$\frac{\d (\Phi\cdot\tgam)}{\d\GA} = \Phi\cdot\frac{\ngam}{r}$$
and from~\eqref{estim00} it follows 
\begin{equation}\label{bound-tang} \|\Phi\cdot\tgam\|_{L^{\infty}(\GA)}\leq \frac{ \|\Phi\cdot\ngam\|_{L^{1}(\GA)}}{r}\,.
\end{equation}
\par Now, as $\Phi\in \Ll_\GA$, we have  
\begin{equation}\label{bound-norm}
\|\Phi\cdot\ngam\|_{L^{2}(\GA)}\leq (1+\rho) \|\Pi_\GA(\nabla_{W^{1,2}(\Ga)}E(\GA))\|_{W^{1,2}(\Ga)}
\end{equation}
and, from~\eqref{bound-tang} and~\eqref{bound-norm}, it follows
\begin{equation}\label{estim0}
 \|\Phi\|_{L^{1}(\GA)}\leq \sqrt{2\pi r}(2\pi +1)(1+\rho) \|\Pi_\GA(\nabla_{W^{1,2}(\Ga)}E(\GA))\|_{W^{1,2}(\Ga)}\,.
 \end{equation}
\par  Concerning the first derivative we remark that, as $\Phi$ is periodic, the mean value of its first derivative is equal to zero. Then, by Poincar\'e's inequality (see Theorem 3.44 in~\cite{AFP}), we have 
\begin{equation}\label{estim1} \left\|\frac{\d \Phi}{\d\GA}\right\|_{L^{1}(\GA)}\leq C_0(r) TV_\GA\left(\frac{\d \Phi}{\d\GA}\right) 
\end{equation}
where $C_0(r)$ is a constant depending on $r$. Moreover, by integrating by parts the integrals of the definition of second variation, we get 
\begin{equation}\label{estim2}
TV^2_\GA(\Phi) =  TV_\GA\left(\frac{\d\Phi}{\d\GA}\right)\,.
\end{equation}
So, in order to prove the lemma it suffices to prove a bound for  $TV_\GA\left(\frac{\d\Phi}{\d\GA}\right)$. 
\par Now, as $\frac{\d \Phi}{\d\GA}=\left(\frac{\d \Phi}{\d\GA}\cdot \ngam\right)\ngam$,  by the generalization of the  product rule  to  $BV$-functions (see Theorem 3.96, Example 3.97, and Remark 3.98 in~\cite{AFP}), we get 
\begin{equation}\label{estim30}
TV_\GA\left(\frac{\d \Phi}{\d\GA}\right)\leq \left(1+\frac{1}{r}\right) TV_\GA\left(\frac{\d \Phi}{\d\GA}\cdot \ngam\right)= \left(1+\frac{1}{r}\right) R_\GA(\Phi) \,,
\end{equation}
where we used the fact that $\ngam$ has no jumps (see Example 3.97 in~\cite{AFP}).

The lemma follows from~\eqref{estim0},~\eqref{estim1},~\eqref{estim2}, and~\eqref{estim30}.
\end{proof}

We can now prove Theorem~\ref{existence1}.

\begin{proof} The proof is based on Lemma~\ref{lem-compacity} and Lemma~\ref{lem-compacity-circle}, so we distinguish two cases: $\GA$ is a circle and it is not.
\par We suppose that $\GA$ is different from a circle. Let $\{\Phi_h\}\subset \Ll_\GA \cap \Cc_\GA$ be a minimizing sequence of  $R_\GA$. We can also suppose $\underset{h}{\sup}\,	R_\GA(\Phi_h)< +\infty$.
From Lemma~\ref{lem-compacity} it follows that 
$$
\underset{h}{\sup}\,	\norm{ \Phi_h }_{BV^{2}(\GA)} \leq 
		C(\GA)\left( (1+\rho) \|\Pi_\GA(\nabla_{W^{1,2}(\Ga)}E(\GA))\|_{W^{1,2}(\Ga)} +
\underset{h}{\sup}\,R_\GA(\Phi_h)\right)\,
$$   
where $C(\GA)$ depends only on $\GA$.  This gives a uniform bound for the $BV^2(\GA)$-norms of $\Phi_h$ and  implies that  $\{\Phi_h\}$  converges (up to a subsequence) toward some  $\Phi \in BV^2(\GA)$ with respect to the $BV^2(\GA)$-weak* topology (see Theorem 3.23 in~\cite{AFP}). 
\par In particular $\Phi_h \rightarrow \Phi$  in $W^{1,1}(\GA)$ which proves that  $\Phi \in \Cc_\GA $, and, by the lower semi-continuity of the $L^2$-norm, we also get  $\Phi \in \Ll_\GA$.
\par Now, as $R_\GA$ is lower semi-continuous with respect to the $BV^2(\GA)$-weak* topology, the theorem ensues.
\par In the case where $\GA$ is a circle with radius $r$, for every minimizing sequence $\{\Phi_h\}\subset \Ll_\GA \cap \Cc_\GA$, we consider the sequence 
\begin{equation}\label{psi-proof}
\Psi_h(s)=\Phi_h(s) - (\Phi_h(s_0)\cdot \tgam(s_0))\tgam(s) \,
\end{equation}
for some $s_0\in \Circ$.
We remark that $\{\Psi_h \} \subset  \Ll_\GA$. 
 Moreover 
\begin{equation}\label{phi-psi}
\frac{\d\Psi_h}{\d\GA}(s)=\frac{\d\Phi_h}{\d\GA}(s)-\left(\frac{\Phi_h(s_0)\cdot \tgam(s_0)}{r}\right)\ngam(s)
\end{equation}
which implies that, for every $h$,  $\Psi_h\in \Cc_\GA$ and
\begin{equation}\label{equal-TV}
R_\GA(\Psi_h)=R_\GA(\Phi_h)\,.
\end{equation}
Then the sequence $\{\Psi_h\}$ is a minimizing sequence of Problem~\eqref{grad-rigid} such that $\Psi(s_0)\cdot\tgam(s_0)=0$. We can also suppose $\underset{h}{\sup}\,	R_\GA(\Psi_h)< +\infty$.

Then, by Lemma~\ref{lem-compacity-circle} we get
$$	 \underset{h}{\sup}\,	\norm{ \Psi_h }_{BV^{2}(\GA)} \leq 
		C(r)\left( (1+\rho) \|\Pi_\GA(\nabla_{W^{1,2}(\Ga)}E(\GA))\|_{W^{1,2}(\Ga)} +
\underset{h}{\sup}\,R_\GA(\Psi_h)\right)\,
$$ 
where $C(r)$ depends only on $r$.


This proves a uniform bound for  $\|\Psi_h\|_{BV^2(\GA)}$ which implies that the minimizing sequence $\{\Psi_h\}$ converges (up to a subsequence) with respect to the $BV^2$-weak* topology.  Then we can conclude as in the previous case.
\end{proof}

We point out that, as showed in the previous proof, when $\GA$ is a circle the Finsler gradient is defined up to a tangential translation. This was actually expected because such a tangential translation is a rotation of the circle.
 
We have defined a penalty for piecewise rigid $BV^2$ deformations for curves belonging to $\Bb$. In the next section we use the Finsler descent method with respect to such a penalty to solve curve matching problems.

%%%%% REMOVED %%%%%
\begin{comment}
\textcolor{red}{deuxieme demo: c'est un peu moins general car on demontre un bound pour une suite minimisante particuliere, mais cela unifie les cas du cercle et non cercle}

\begin{lem}\label{lem-compacity-psi}
Let  $\GA\in \mathcal{B}$  be an injective curve. For every  $\Phi \in \Ll_\GA \,\cap\, C^2(\GA)$ there exists a unique $(a,b)\in\R\times\RR^2$ such that  
$$\Psi=\Phi +a \GA^\bot +b$$
verifies 
\begin{equation}\label{conditions}
(\Psi\cdot\tgam)(s_0)=(\Psi\cdot\ngam)(s_0)=\left(\frac{\d \Phi}{\d\GA}\cdot \ngam\right)(s_0)=0
\end{equation}
for some $s_0\in \Circ$. In particular $\Psi\in\Cc_\GA$ and we have 
$$TV_\GA\left(\frac{\d \Psi}{\d\GA}\cdot \ngam\right)=TV_\GA\left(\frac{\d \Phi}{\d\GA}\cdot \ngam\right)\,.$$
Moreover there exists a constant  $C$ depending on $\GA$ such that 
	\begin{equation}\label{compactness-psi2}
	 	\norm{ \Psi }_{W^{2,1}(\GA)} \leq 
		C\left( 
TV_\GA\left(\frac{\d \Phi}{\d\GA}\cdot \ngam\right)\right)(1+|D^2\GA|(\Circ))\, .
	\end{equation}

\end{lem}

\begin{proof} Considering a deformation  $\Psi=\Phi +a \GA^\bot +b$ we get
$$\Psi\cdot\tgam = \Phi\cdot\tgam + a\GA^\bot\cdot\tgam + b\cdot\tgam $$ 
$$\Psi\cdot\ngam = \Phi\cdot\ngam + a\GA^\bot\cdot\ngam + b\cdot\ngam $$ 
$$\frac{\d \Psi}{\d\GA}=\frac{\d \Phi}{\d\GA} + a \ngam $$
which determines the unique $(a,b)$ such that $\Psi$ verifies~\eqref{conditions}. In particular, given $s_0\in\Circ$ we have 
$$a=-\left(\frac{\d \Phi}{\d\GA}\cdot\ngam \right)(s_0) $$
$$b=-(\Phi\cdot\tgam + a\GA^\bot\cdot\tgam)(s_0)\tgam - (\Phi\cdot\ngam + a\GA^\bot\cdot\ngam)(s_0)\ngam\,.$$
Moreover, as $\Phi\in\Cc_\GA$ and $\frac{\d \Psi}{\d\GA}=\frac{\d \Phi}{\d\GA} + a \ngam$, we get that 
$\Psi\in\Cc_\GA$ and we have 
$$TV_\GA\left(\frac{\d \Psi}{\d\GA}\cdot \ngam\right)=TV_\GA\left(\frac{\d \Phi}{\d\GA}\cdot \ngam\right)\,.$$
We prove now   estimate~\eqref{compactness-psi2}. In the following we denotes by $C$ a generic constant depending on $\GA$.
Now, as $\Psi$ is periodic, the mean value of  its first derivative is equal to zero, so by Poincar\'e's inequality we have 
\begin{equation}\label{norme-infty-psi}
\left\|\frac{\d \Psi}{\d\GA}\right\|_{L^{\infty}(\GA)}=\left\|\frac{\d \Psi}{\d\GA}\cdot\ngam\right\|_{L^{\infty}(\GA)}\leq C TV_\GA\left(\frac{\d \Phi}{\d\GA}\cdot \ngam\right)\,.
\end{equation}
Moreover, by~\eqref{conditions} and  the fact that $\Psi\in\Cc_\GA$ and $\Psi(s_0)=0$, we get 
\begin{equation}\label{norme-infty-psi-2}
\norm{ \Psi}_{L^\infty(\GA)} \leq C TV_\GA(\Psi) = C \left\|\frac{\d \Psi}{\d\GA}\cdot\ngam\right\|_{L^{1}(\GA)}\,.
\end{equation}
Then  by~\eqref{norme-infty-psi} and~\eqref{norme-infty-psi-2}, we get 
\begin{equation}\label{bound-first-der-psi}
\norm{ \Psi}_{W^{1,\infty}(\GA)}\leq C
TV_\GA\left(\frac{\d \Phi}{\d\GA}\cdot \ngam\right)\, . 
\end{equation}
Concerning the second derivative, as $L_\GA(\Psi)=0$, we get
$$ 0 = D_\GA\left( \frac{\d\Psi}{\d\GA}\cdot \tgam\right) =\left( \frac{\d^2\Psi}{\d\GA^2}\cdot \tgam \right)\d\GA(s) + \left( \frac{\d\Psi}{\d\GA}\cdot \ngam\right)\d \rm{curv}_\GA(\GA)\,$$
and
$$ 
D_\GA\left(\frac{\d \Psi}{\d\GA}\cdot\ngam\right)=\left(\frac{\d^2 \Psi}{\d \GA(s)^2}\cdot\ngam\right)\d\GA(s)\,.$$
Then we have
\begin{equation}\label{bound-second-der-psi} 
\begin{array}{ll}
\displaystyle{\left\|\frac{\d^2 \Psi}{\d \GA(s)^2}\right\|_{L^{1}(\GA)}}&\displaystyle{\leq \left\|\frac{\d^2 \Psi}{\d \GA(s)^2}\cdot\tgam\right\|_{L^{1}(\GA)} + \left\|\frac{\d^2 \Psi}{\d \GA(s)^2}\cdot\ngam\right\|_{L^{1}(\GA)}}\\
&\displaystyle{\leq   |\rm{curv}_\GA(\GA)|(\Circ)\norm{ \Psi\cdot \ngam}_{L^\infty(\GA)}+TV_\GA\left(\frac{\d \Psi}{\d\GA}\cdot\ngam\right)
}\\
&\displaystyle{\leq   |D^2\GA|(\Circ)\norm{ \Psi}_{L^\infty(\GA)}+TV_\GA\left(\frac{\d \Psi}{\d\GA}\cdot\ngam\right)
}
\,.
\end{array}
\end{equation}
Estimate~\eqref{compactness-psi2} follows by~\eqref{bound-first-der-psi} and~\eqref{bound-second-der-psi}.
\end{proof}

\begin{proof} (of {\bf Theorem~\ref{existence1}}) Let $\{\Phi_h\}\subset \mathcal{B}$ be a minimizing sequence of  $R_\GA$. Then, we define the sequence
$$\Psi_h=\Phi_h+ d_h\varphi_h\,$$
where 
$$
d_h=\|(a_h\GA^\bot+b_h)\cdot\ngam\|_{L^2(\GA)}\,,\quad \varphi_h = \frac{a_h\GA^\bot+b_h}{\|(a_h\GA^\bot+b_h)\cdot\ngam\|_{L^2(\GA)}}
$$
where $a_h,b_h$ are given  by Lemma~\ref{lem-compacity-psi}. \par We remark that, because of the approximation theorem for $BV$-function with respect to the $BV$-strict topology (see Theorem 3.9 in~\cite{AFP}),~\eqref{compactness-psi2} can be relaxed to $BV^2(\GA)$. It follows that 
	\begin{equation}\label{compactness-psi2bis}
	 	\norm{ \Psi_h }_{W^{2,1}(\GA)} \leq 
		C\left( 
TV_\GA\left(\frac{\d \Phi_h}{\d\GA}\cdot \ngam\right)\right)(1+|D^2\GA|(\Circ))\, 
	\end{equation}
where $C$ depends on $\GA$.
\par For every $h$, we aim to solve the following problem
\begin{equation}\label{problem2}
\rm{Min}\,\{TV_\GA(\alpha\Psi_h+\beta\varphi_h)\,|\,\alpha,\beta \in \R,\,\alpha\Psi_h+\beta\varphi_h\in \Ll_\GA \}\,.
\end{equation}
We remark that for every $h$ and $\alpha,\beta \in \R$ $$TV_\GA(\alpha\Psi_h+\beta\varphi_h)=|\alpha|TV_\GA(\Psi_h)\,$$
and that the minimization problem is well-posed because $\Phi_h=1\Psi_h-d_h\varphi_h$ belongs to $\Ll_\GA$. This means in particular that a solution to problem~\eqref{problem2} verifies $|\alpha| \leq 1$. In the following we assume that $|\alpha| \leq 1$.
\par Now, the constraint $\alpha\Psi_h+\beta\varphi_h\in \Ll_\GA$ implies that
\begin{equation}\label{bilinear2}
(\alpha, \beta)\cdot A
\begin{pmatrix}
\alpha\\
\beta\\
\end{pmatrix}
\leq
(1+\rho) \|\Pi_\GA(\nabla_{L^2(\GA)}E(\GA))\|_{L^{2}(\GA)}
\end{equation}
where the matrix $A$ is defined by
\begin{equation}
\begin{pmatrix}
\displaystyle{\int_{\Circ} \left(\Psi_h\cdot\ngam\right)^2\d \GA(s)}&\displaystyle{\int_{\Circ}\left(\Psi_h\cdot\ngam\right)\left(\varphi_h\cdot\ngam\right) \d \GA(s)}\\
\displaystyle{\int_{\Circ}\left(\Psi_h\cdot\ngam\right)\left(\varphi_h\cdot\ngam\right)\d \GA(s)}&\displaystyle{\int_{\Circ} \left(\varphi_h\cdot\ngam\right)^2\d \GA(s)}\\
\end{pmatrix}
.
\end{equation}
We suppose that such a bilinear form is not degenerate.
\par Now,  $\|\Pi_\GA(\varphi_h)\|_{L^2(\GA)}=1$ for every $h$, so that by solving inequality~\eqref{bilinear2}, as $|\alpha|\leq 1$ and  because of~\eqref{compactness-psi2}, we have that $\beta$ can be bounded by a constant depending on $\GA$ and $R_\GA(\Phi_h)$.

\par If the previous bilinear form is degenerate we have 
$$
(\Psi_h\cdot\ngam)(s)=(\varphi_h\cdot\ngam)(s) \quad \forall s \in \Circ\,.
$$ 
In particular, because of the definition of $\Ll_\GA$ we get that $\varphi_h\cdot\ngam$ is not identically zero. In fact, if $\varphi_h\cdot\ngam$ was identically zero then $(\alpha\Psi_h+\beta\varphi_h)\cdot\ngam$ would be identically zero which is in contradiction with the fact that $\alpha\Psi_h+\beta\varphi_h\in \Ll_\GA$. 
\par Thus, in the degenerate case, we get that $\alpha\Psi_h+\beta\varphi_h\in \Ll_\GA $ is equivalent to $(\alpha+\beta)\varphi_h\in \Ll_\GA $, so that 
\begin{equation}\label{ab-bound}\|\Pi_\GA((\alpha+\beta)\varphi_h-\nabla_{L^2(\GA)}E(\GA))\|_{L^{2}(\GA)}\leq \rho
\|\Pi_\GA(\nabla_{L^2(\GA)}E(\GA))\|_{L^{2}(\GA)}\,.
\end{equation}
Now, as  $\|\Pi_\GA(\varphi_h)\|_{L^{2}(\GA)}=1$ for every $h$ and $|\alpha|<1$,~\eqref{ab-bound} proves that $\alpha$ and $ \beta$ are bounded by a constant depending only on $\GA$. This in particular guarantees the existence of a solution  to Problem~\eqref{problem2}. 
\par Finally we have proved that for every $h$, Problem~\eqref{problem2} admits at least a  solution, denoted by  $(\alpha_h, \beta_h)$.  In particular $\alpha_h$ and $\beta_h$ can be bounded by a constant depending on $\GA$ and $R_\GA(\Phi_h)$.
\par We consider now the sequence
$$\tilde{\Phi}_h = \alpha_h\Psi_h+\beta_h\varphi_h$$
where $(\alpha_h, \beta_h)$ is a solution of Problem~\eqref{problem2}. We have 
$$TV_\GA(\tilde{\Phi}_h)=|\alpha|TV_\GA(\Psi_h)\leq TV_\GA(\Phi_h)$$
because $\Phi_h=1\Psi_h-d_h\varphi_h$ belongs to $\Ll_\GA$.
Moreover $\tilde{\Phi}_h\in \Cc_\GA\cap \Ll_\GA$. Then  $\{\tilde{\Phi}_h\}$ defines a new minimizing sequence for Problem~\eqref{grad-rigid}.
 Moreover, as said above $\alpha_h, \beta_h$ are bounded by a constant depending on $\GA$ and $R_\GA(\Phi_h)$ and  $\|\Pi_\GA(\varphi_h)\|_{L^2(\GA)}=1$ for every $h$. \textcolor{red}{il faut trouver une borne superieure pour la norme L2 de $\varphi\cdot\tgam$ i.e. pour $\Phi\cdot\tgam$} This implies that 
$$
\underset{h}{\sup}\,	\norm{ \tilde{\Phi}_h }_{BV^{2}(\GA)} \leq 
		C\left( (1+\rho) \|\Pi_\GA(\nabla_{L^2(\GA)}E(\GA))\|_{L^{2}(\GA)} +
TV_\GA\left(\frac{\d \Phi}{\d\GA}\cdot \ngam\right)\right)(1+|D_\GA^2\GA|(\Circ))\,.
$$   
\par This gives an uniform bound for the $BV^2(\GA)$-norms of $\Phi_h$ and  implies that  $\{\Phi_h\}$  converges (up to a subsequence) toward some  $\Phi \in BV^2(\GA)$ with respect to the $BV^2(\GA)$-weak* topology (see Theorem 3.23 in~\cite{AFP}). 
\par In particular $\Phi_h \rightarrow \Phi$  in $W^{1,1}(\GA)$ which proves that  $\Phi \in \Cc_\GA $, and, by the lower semi-continuity of the $L^2$-norm, we also get  $\Phi \in \Ll_\GA$.
\par Now, as $R_\GA$ is lower semi-continuous with respect to the $BV^2(\GA)$-weak* topology, the theorem ensues.
\end{proof}
\end{comment}
%%%%% END REMOVED %%%%%






